% arara: pdflatex
% arara: pdflatex
% --------------------------------------------------------------------------
% the LIBERTINEHOLOGOPATCH package
% 
%   patch kerning of hologo's logos for usage with the `libertine' package
% 
% --------------------------------------------------------------------------
% Clemens Niederberger
% Web:    https://bitbucket.org/cgnieder/libertinehologopatch/
% E-Mail: contact@mychemistry.eu
% --------------------------------------------------------------------------
% Copyright 2013-2015 Clemens Niederberger
% 
% This work may be distributed and/or modified under the
% conditions of the LaTeX Project Public License, either version 1.3
% of this license or (at your option) any later version.
% The latest version of this license is in
%   http://www.latex-project.org/lppl.txt
% and version 1.3 or later is part of all distributions of LaTeX
% version 2005/12/01 or later.
% 
% This work has the LPPL maintenance status `maintained'.
% 
% The Current Maintainer of this work is Clemens Niederberger.
% --------------------------------------------------------------------------
% The libertinehologopatch package consists of the files
%  - libertinehologopatch.sty
%  - libertinehologopatch_en.tex, libertinehologopatch_en.pdf
%  - README
%  - testfiles: testfile.tex, testfile-xe.tex, testfile-lua.tex
% --------------------------------------------------------------------------
% If you have any ideas, questions, suggestions or bugs to report, please
% feel free to contact me.
% --------------------------------------------------------------------------
% \PassOptionsToPackage{supstfm=libertinesups}{superiors}
\documentclass[load-preamble+]{cnltx-doc}
\usepackage{libertinehologopatch}
\newpackagename\libertinehologopatch{libertinehologopatch}
\setcnltx{
  title    = {libertine\-hologo\-patch},
  package  = {libertinehologopatch},
  info     = patch kerning of logos provided by \pkg*{hologo} for use with
    \pkg*{libertine} ,
  url      = https://bitbucket.org/cgnieder/libertinehologopatch/ ,
  authors  = Clemens Niederberger ,
  email    = contact@mychemistry.eu ,
  subtitle = \huge from {\restorelogos\LaTeXTeX} to \LaTeXTeX ,
  abstract = {%
    Patching the logos provided by the \pkg*{hologo} package when used
    together with the \pkg*{libertine} package and \pdfLaTeX, \XeLaTeX\ or
    \LuaLaTeX.
  } ,
  add-cmds = {
    LaTeX, LaTeXe, LaTeXTeX,
    LuaLaTeX, LuaTeX,
    pdfLaTeX, pdfTeX,
    restorelogos,
    TeX,
    XeLaTeX, XeTeX
  } ,
  % add-silent-cmds = {
  % 
  % } ,
  index-setup = { othercode=\footnotesize, level=\section , noclearpage } ,
  makeindex-setup = {  columns=3, columnsep=1em }
}

\begin{document}

\section{License and Requirements}
\license

\libertinehologopatch\ loads and needs the packages
\pkg{etoolbox}~\cite{pkg:etoolbox}, \pkg{hologo}~\cite{pkg:hologo},
\pkg{ifxetex}~\cite{pkg:ifxetex}, \pkg{ifluatex}~\cite{pkg:ifluatex} and
\pkg{libertine}~\cite{pkg:libertine}.

\section{Options}
There are a few package options:
\begin{options}
  \opt{verbose}
    Inform when patching is successful and warn if it fails. This is the
    default behaviour.
  \opt{silent}
    Neither inform nor warn if patching succeeds or fails.
  \opt{errors}
    Issue an error instead of a warning if patching fails.
  \opt{shortcuts}
    (Re-) Define shortcuts for common logos to use the newly patched
    \pkg{hologo} logos: \cs{TeX} \TeX, \cs{LaTeX} \LaTeX, \cs{LaTeXe} \LaTeXe,
    \cs{LaTeXTeX} \LaTeXTeX, \cs{pdfTeX} \pdfTeX, \cs{pdfLaTeX} \pdfLaTeX,
    \cs{XeTeX} \XeTeX, \cs{XeLaTeX} \XeLaTeX, \cs{LuaTeX} \LuaTeX,
    \cs{LuaLaTeX} \LuaLaTeX. This is default behaviour.
  \opt{noshortcuts}
    Don't (re-) define above mentioned logos.
\end{options}

\section{Usage}
The \pkg{hologo} package by Heiko Oberdiek provides a comprehensive list of
and access to the most common \TeX-related logos. However, the kerning values
it uses are optimized for use with \TeX's standard font, Computer Modern. I
happen to like the Linux Libertine~O and Linux Biolinum~O fonts very
much. They are used in this documentation as well. Unfortunately many of the
logos look bad with these fonts to say the least:

\begin{example}
  \restorelogos\huge
  \LaTeX, \XeLaTeX, \LaTeXTeX \par
  \textit{\LaTeX, \XeLaTeX, \LaTeXTeX}
 
  \sffamily
  \LaTeX, \XeLaTeX, \LaTeXTeX \par
  \textit{\LaTeX, \XeLaTeX, \LaTeXTeX}
\end{example}

What \libertinehologopatch\ does is it checks if one of these fonts is used
and changes the kerning then:

\begin{example}
  \huge
  \LaTeX, \XeLaTeX, \LaTeXTeX \par
  \textit{\LaTeX, \XeLaTeX, \LaTeXTeX}
 
  \sffamily
  \LaTeX, \XeLaTeX, \LaTeXTeX \par
  \textit{\LaTeX, \XeLaTeX, \LaTeXTeX}
\end{example}

All you have to do is load the package in the usual \LaTeXe\ way:
\begin{sourcecode}
  \usepackage{libertinehologopatch}
\end{sourcecode}
and all the logos will be patched appropriately.

\libertinehologopatch\ defines additional user macros:
\begin{commands}
  \command{restorelogos}
    Restores the original definitions of the logo commands, the scope is
    local.
  \command{originallogo}[\marg{logo}]
    This command takes the same argument as \cs{hologo} and outputs the
    unpatched version of the respective logo. It uses \cs{restorelogos}
    internally.
  \command{Originallogo}[\marg{logo}]
    This command takes the same argument as \cs{Hologo} and outputs the
    unpatched version of the respective uppercased logo. It uses
    \cs{restorelogos} internally.
\end{commands}

\section{Implementation}
\implementation{libertinehologopatch.sty}

\cleardoublepage

\end{document}
